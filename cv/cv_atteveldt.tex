% Compile using
% rm cv.bbl ; latex cv && bibtex cv && python processbbl.py && latex cv && latex cv && dvipdf cv
% to use post-processing of bibiliography in processbbl

% to add
% keynote regional
% editorial board media psych
% PC 7th International Conference on Computational Social Science IC 2 S 2

\documentclass[a4paper,11pt]{article}

% Set of commands used by cv.tex

\usepackage[dvipsnames,usenames]{color}
\usepackage{simplemargins}
\usepackage{graphicx}
\usepackage{soul}
\usepackage{tweaklist}
\usepackage{charter}
\usepackage[scaled]{helvet}
\usepackage{sectsty}
\usepackage{engord}
\usepackage{url}

%\usepackage{natbib}
%\def\bibsection{\subsection*{\refname}} 

\sectionfont{\normalfont\blocktitle}
\subsectionfont{\normalfont\pubsect}
\setcounter{secnumdepth}{0}
\setleftmargin{3.0cm}
\setrightmargin{3.0cm}
\settopmargin{3cm}
\setbottommargin{0cm}
\newcommand{\nth}[1]{\engordnumber{#1}}

% small caps
\newcommand{\smcp}[1]{\sc{\scriptsize{#1}}}
\renewcommand{\textsc}[1]{\smcp{#1}}

\newcommand{\header}[2]{
\begin{centering} 
{\textsc{\so{CURRICULUM VITAE}}}
\\\vspace*{5mm}
{\textsf{\large{#1}}}
\\\vspace*{5mm}
{\textsc{\so{#2}}}

\end{centering}
}


\newcommand{\bold}[1]{{\textbf{#1}}}
\newcommand{\italics}[1]{{\textit{\footnotesize #1}}}

\newcommand{\blocktitle}[1]{%
\textcolor{MidnightBlue}{\textsf{\large \hspace{1.4mm}#1}}
\vspace{1mm}
\hrule\par\mbox{}
}

\newcommand{\pubsect}[1]{%
\vspace{-1em}\textcolor{MidnightBlue}{\textsf{\normalsize\center #1}}
\vspace{1mm}
%\hrule\par\mbox{}
}

\newcommand{\yearhead}[2]{\parbox[t]{4cm}{\textsf{\small #1}} \parbox[t]{13cm}{\textsf{#2}}}
\newcommand{\body}[1]{\mbox{}\hspace{4.5cm}\parbox[t]{10.5cm}{\small #1}}

\newcommand{\resumeitem}[3]{\yearhead{#1}{\bold{#2}}\\*\body{#3}\vspace{.3em}}
\newcommand{\simpleresumeitem}[2]{\yearhead{#1}{#2}}
\newcommand{\resumeitemize}[3]{\resumeitem{#1}{#2}{\vspace{-1em}\begin{itemize}#3\end{itemize}}}


\newcommand{\bodyb}[1]{\mbox{}\hspace{1.5cm}\parbox[t]{13.5cm}{\small #1}}
\newcommand{\resumeitemb}[2]{\bold{\textsf{#1}}\\\bodyb{#2}\vspace{.3em}}
\newcommand{\resumeitemizeb}[2]{\resumeitemb{#1}{\vspace{-1em}\begin{itemize}#2\end{itemize}}}
%%%%%%

\setlength{\parindent}{0mm}
\pagestyle{empty}

\renewcommand{\itemhook}{\setlength{\topsep}{0pt}%
  \setlength{\itemsep}{-0mm}\setlength{\leftmargin}{1em}}
\renewcommand{\labelitemi}{-}

\usepackage[left=3cm,top=1.5cm,right=3cm,bottom=1.5cm]{geometry}
\usepackage{eurosym}
\usepackage{hyperref}
\usepackage[backend=biber,url=true,style=apa,sorting=ydnt]{biblatex}


\hyphenation{Loecher-bach}

\begin{document}


\begin{center}
\textit{Curriculum Vitae}

Wouter van Atteveldt, September 9, 2022
\end{center}
%\setmainfont


%\header{Wouter van Atteveldt, Ph.D.}{April 11, 2013}

%%%%%%%%%%%%%%%%%%%%%%%%%%%%%%%%%%%%%%%%%%%%%%%%%%%%%%%%%%%%%%%%%%%%%%%%%%%%%%%%


\newcommand{\details}{
\section{Personal Details}

\simpleresumeitem{Name}{Dr. Wouter van Atteveldt}
%\simpleresumeitem{Date of Birth}{10 March 1980}
%\simpleresumeitem{Citizenship}{Dutch}
\simpleresumeitem{Address}{Brouwersgracht 935, 1015 GK Amsterdam, The Netherlands}
%\simpleresumeitem{Telephone}{+31 6 42740725}
\simpleresumeitem{E-mail}{wouter@vanatteveldt.com}
\simpleresumeitem{Home page}{http://vanatteveldt.com}
\simpleresumeitem{Current}{Professor of Computational Communication Science \& \\ Political Communication, VU University Amsterdam}
}

%%%%%%%%%%%%%%%%%%%%%%%%%%%%%%%%%%%%%%%%%%%%%%%%%%%%%%%%%%%%%%%%%%%%%%%%%%%%%%%%

\newcommand{\education}{
\section{Post-secondary Education}


\resumeitem{2004-2008}{Ph.D. in Artificial Intelligence and Communication Science}
{VU University Amsterdam, VUBIS Interdisciplinary Ph.D. program\\
Department of Exact Sciences and Department of Social Sciences\\
Advisors: prof. Frank van Harmelen (AI), prof. Jan Kleinnijenhuis (Communication Science), dr. Stefan Schlobach (AI)
}

\resumeitem{2002 -- 2003}{Master of Science in Informatics}
                       {University of Edinburgh
                       \\Specialisation: Language Engineering}

\resumeitem{1998 -- 2001}{Bachelor of Science (\italics{cum laude})}
                    {University College Utrecht
                    \\Major: Science (Mathematics, Computer Science)
                    \\Minor: Humanities (Linguistics)}
}


%%%%%%%%%%%%%%%%%%%%%%%%%%%%%%%%%%%%%%%%%%%%%%%%%%%%%%%%%%%%%%%%%%%%%%%%%%%%%%%%

\newcommand{\work}{
\section{Relevant Professional experience}

\resumeitem{2022 -- present}
{VU University Amsterdam: Full Professor}
{Faculty of Social Sciences, Department of Communication Sciences}

\resumeitem{2017 -- 2021}
{VU University Amsterdam: Associate Professor}
{Faculty of Social Sciences, Department of Communication Sciences}

\resumeitem{2010 -- 2017}
{VU University Amsterdam: Assistant Professor}
{Faculty of Social Sciences, Department of Communication Sciences}

\resumeitem{2017 January -- May}
{Visiting Professor, City University of Hong Kong}
{Department of Media and Communication\\
Developing automatic methods for analysing social media (hosted by Jonathan Zhu)
}

\resumeitem{2016 -- present}
{Faculty Affiliate, Cline Center for Democracy}{University of Illinois Urbana-Champaign}

\resumeitem{2013 February -- July}
{Visiting Scholar, Hebrew University in Jerusalem}
{Department of Political Science\\
Using automated content analysis to study Mediated Public Diplomacy (hosted by Tamir Sheafer and Shaul Shenhav)
}

\resumeitem{2009 -- 2012}
{University of Antwerp: Researcher (0.2 FTE)}
{Department of Political Science
\\ Using machine learning to automate Agenda coding of newspaper articles, legislative texts, and manifestoes
}

\resumeitem{2008 -- 2010}
{VU University Amsterdam: Post-doc Researcher}
{Faculty of Social Sciences, Department of Communication Sciences
\\ Research in NWO program \italics{Omstreden Democratie} (Contested Democracy) on the effects of short term media logic on long term democratic viability
}


\resumeitem{2004 -- 2008}
{VU University Amsterdam: Promovendus (PhD student)}
{Faculty of Sciences, Department of Artificial Intelligence and Faculty of Social Sciences, Department of Communication Sciences
\\ Dissertation defended 14th of November, 2008
%\\ Organised weekly AI seminar and two departmental retreats
}

\resumeitem{March -- June 2006}
{Carnegie Mellon University, Visiting Researcher}
{Using Social Network Analysis methods to analyse political data \\
Hosted by Kathleen Carley
}

\resumeitem{2001 -- 2006}
{2AT: Co-founder, programmer, director}
{Custom software solutions, network service, system administration.}
}

%%%%%%%%%%%%%%%%%%%%%%%%%%%%%%%%%%%%%%%%%%%%%%%%%%%%%%%%%%%%%%%%%%%%%%%%%%%%%%%%

\newcommand{\summary}{
\section{Brief Summary of Research}
  
Due to my interdisciplinary background in Computational Linguistics and Communication Science, I have acquired a
leading position in the budding field of computational communication science, specializing
in the automatic analysis of large collections of political texts.
I acquired various grants (NWO VENI, OC; Digging into Data; Joint E-science/Data Science; ERC H2020, Horizon RIA) to study
susbtantive and methodological aspect of political communication. 
This
resulted in a number of substantive collaborations and articles in the best-cited journals in the
field, including Political Analysis, Journal of Communication, and Political Communication. I currently have an h-index of 24 with over 3,400 citations (Google Scholar).

\hspace{2em} These techniques have been disseminated to a broad audience of social science scholars,
as R packages and as part of the AmCAT text analysis infrastructure that I developed during my PhD project and that
is used in teaching and research projects at various universities. As an indication of the use of
this   software   in   research   and   teaching,   in  2016   alone   442   unique   users  used AmCAT in 494 different projects,
uploading 8,719,400 documents and conducting 9,768 queries.
AmCAT has been cited over 100 times in academic work. 

\hspace{2em} Substantively I contributed to political communication, especially together with the Ph.D.
students I supervise(d): supervised machine learning (Moritz Laurer) news algorithms and diversity (Nicolas Mattis); digital tracking and filter bubbles (Felicia Loecherbach); the role of media logic in recent election campaigns (Janet Takens); the
consequences of increased audience logic and the gatekeeping role of newspapers (Kasper Welbers),
and tabloidization and news quality  (Carina Jacobi). 

\hspace{2em} My   interdisciplinary   background   puts   me   in   a   unique   position   to   continue   to   push   the
methodological state of the art in text analysis.
As part of this effort, I co-founded and chaired (until 2020) the Computational Methods interest group in the International Communication Association (ICA),
and co-founded and chair the CCS.Amsterdam virtual collaborative research lab. 
I am also editor-in-chief and co-founder of the Q1-ranked open access journal Computational Communication Research.
}

\newcommand{\teaching}{
\section{Other Relevant Experience}

\resumeitemizeb{Ph.D. Supervision}{
\item Zeve Sanderson (promotor; external), started Septemer 2022 on \textit{Belief in online (mis)information}
\item Moritz Laurer (promotor; external), will defend Octover 2024 on \textit{Transfer Learning for Computational Social Science}
\item Nicolas Mattis (promotor), will submit September 2024 on \textit{Rethinking News Algorithms}
\item Felicia Loecherbach (promotor), defended 2023 on \textit{Tracking the Filter Bubble}, 
\item Christian Erwich (co-promotor) defended 2021  on using natural language techniques  for interpreting bible texts (promotor: prof. Wido van Peursen).
\item Kasper Welbers (co-promotor and daily supervisor), defended his thesis on Gatekeeping in the Digital Age on 9 December 2016 (promotor: prof. Jan Kleinnijenhuis)
\item Carina Jacobi (co-promotor), defended her dissertation on the quality of political  news in a changing media  environment at the University of  Amsterdam on  January 22 nd , 2016  (promotor: prof. Klaus Schonbach).
\item Janet Takens (co-promotor), successfully defended her dissertation on
 \textit{Media Logic and Electoral Democracy} on  February 1st, 2013 (promotor: prof. Jan Kleinnijenhuis)
\item Numerous M.A. and  B.A. theses in Communication and Journalism
}

% \newpage

% \resumeitemizeb{Invited presentations and Keynotes}{
% \item November 2020: Invited speaker, University of Bern (online)
% \item September 2020: Visiting fellow, Fraunhofer Institute Berlin (postponed)
% \item November 2019: Invited speaker, University of Mannheim
% \item May 2019: Invited speaker, University of Stavanger
% \item April 2019: Invited speaker, University of Z\"urich
% \item April 2019: Paul Lazarsfeld Guest Professor and Invited Speaker, University of Vienna
% \item October 2018: Invited speaker, University of Salamanca
%   \item November 2018: Invited participant, University of Bergen
% \item April 2018: Invited participant, Text Analysis Developer Workshop, NYU, April 20-21
% \item January 2018: Invited participant, Collaborative Research on Extreme-Scale Text Analytics (CRESTA) workshop, UIUC, January 1-4
% \item May 2017: Invited speaker at Kobe University (Japan)
% \item February 2017: Invited speaker at City University of Hong Kong, COM Research Seminar 
% \item June 2016: Invited speaker at NetGlow conference, St. Petersburg
% \item April 2016: Invited speaker in UvA Text Analysis series
% \item March 2016: Invited speaker at Text Visualization workshop, LSE and Imperial College, London
% \item March 2015: invited speaker at Analyzing Political Discourse in the International Arena, Hebrew University, Israel
% \item February 2015: invited speaker at Innovations in Quantitative Content Analysis workshop , European University Institute, Florence
% \item August 2015: invited speaker at ‘Network Theory and Methods' conference (VU University)
% \item October 2015: invited speaker at Connecting Data for Research symposium (VU University)
% \item October 2015: invited speaker at Talk of Europe workshop organized by Royal Library (KB), The Hague               
% \item Dec 2014:   Keynote   speaker   at   Daegu   Gyeongbuk   International   Social   Network Conference organized by Asia, Triple Helix Society, Daegu (Korea).
% \item Nov 2014: Invited speaker at SFB 884 Text Conference, MZES Mannheim.
% \item Jun 2014: Invited speaker at Big Data in the Social Sciences, University of Glasgow. 
% \item Jul 2013: Invited speaker at New Directions in Analyzing Text as Data, LSE London.
% }

% \resumeitemizeb{Workshop Organization}{
% \item December 2018: co-organized EuroCSS workshop on Workshop on Biases in Social Computing Data and Technology
% \item June 2014 and June 2015: co-organized pre-conference workshops on social and
% semantic   network   methods   for   communication   science   at   the   International
% Communication Association (ICA; with Jana Diesner and Christian Baden).
% \item January   2013:   co-organized    workshop   on   Media
% Logic and Electoral Democracy at VU University (with Janet Takens).
% \item Nov 2008: Organized workshop on Semantic Network Analysis at VU University 
% }

% \resumeitemizeb{Tool Development}
% {\item AmCAT: I developed the open source Amsterdam Content Analysis Toolkit (AmCAT) 
% for media storage, retrieval, annotation, and analysis; currently leading team of
% 2 developers for maintaining a system containing 40 million documents,
% of which over 400 thousand manually annotated, in use by $>$100 users
% at various institutions and companies. See
% \url{http://github.com/amcat} and \url{http://amcat.nl}.
% AmCAT has been cited in over 100 articles.
% \item R packages: I (co-)developed various R packages, including RTextTools (machine learning), Corpustools (topic modeling), RNewsflow (document similarity)
% \item Natural Language Processing: I have developed nlpipe, a toolkit for easily processing large amounts of text with NLP tools, integrated with AmCAT and R.  
% }


% \resumeitemizeb{Methodological Dissemination} {
% \item 2020: workshop on Analyzing Text and Data with R, University of Bern
% \item 2019, 2020: workshop on Topic Modeling, GESIS Cologne
% \item 2019: hands-on workshop on deep learning, University of Mannheim 
% \item 2019: workshop on Analyzing Text and Data with R, University of Vienna
% \item 2018: workshop on Analyzing Text and Data with R, FSW Graduate School
% \item 2016: workshop on Text Analysis with R, University of Glasgow, 17 November 2016
% \item 2016: developed and taught summer school course on Advanced Text Analysis with R, City University Hong Kong (Ph.D. level, 6x3 hours)
% \item 2015 -- 2019: co-organized and taught summer and winter school on Big Data, VU University
% \item 2014 -- 2016: organized and taught `Programming and Analyzing in R`, VU Graduate program 
% \item Numerous workshops on text analysis and AmCAT, including at St. Petersburg University (2016), University of Ghent
% (2015), Vienna (2014), Antwerp (2014), HU Jerusalem (2014), Washington (2013),
% UT Austin (2012)
% }

\resumeitemizeb{Course Co-ordinator, developer and lecturer}{
\item 2019-2022: Programme Director of VU master and bachelor program Communication Science. Responsible for BA and MSc curriculum review, introduction of computational methods, and development of international track. 
\item developed and taught M.Sc. level course `Political Communication and News Dynamics'; co-ordinated and taught M.Sc. level course `News effects in the Digital Age' 
\item developed and taught B.Sc. level courses `Big Data and Computational Thinking`, `Social Media Analytics', `Seminar on Public Opinion and Communication', `Framing in Politics and Economics', and `Media in the Public Domain'
\item lecturer and guest lecturer in various courses including `Current Issues in Communication Science' (M.Sc.) and `Marketing Communication 2.0' (B.Sc.)
}
\newpage
\resumeitemizeb{University Organization and Community Service}{
  \item Director, FSW Societal Analytics Lab
\item Member of the FSW Research Assessment Board (2021 -- 2023)
\item Member of the Programme Committee for the Research Master programme. 
\item Co-founder and Chair of the ICA Division on Computational Methods, ICA\\ (2016 -- 2020)
\item Co-founder and Chair of CCS.Amsterdam (2018 -- present)
\item Member of the departmental and central examination committees for
  Communication Science, VU Faculty of Social Sciences, and Journalism
\item Chair of VU FSW promotion committees (2019 \& 2020)
\item Member of the VU FSW committees on testing and evaluation; research Master; and methods and techniques. 
\item Member of the jury for the Faculty Research Prize (2013-2020)
}

\resumeitemizeb{Reviewing and editing}{
\item Founding editor-in-chief, Computational Communication Research (ISSN 2665-9085, Scimago Q1).
\item Guest editor, Special Issue on Computational Methods, Communication Methods and Measures
\item Editor for Journal of Communication (2022 -- present)
\item Editor for Journal of Media Psychology (2022 -- present)
\item Editor for Communication Methods and Measures (2019 -- present)
\item Editor for the Journal of Information Technology and Politics (2011 -- present)
\item Program Committee Member for the International Conference on Computational Social Science (2019 -- present), Web Science conference (2011 --  2013), IEEE eScience conference (2018), LREC (2020)
\item Reviewer for e.g. Journal of Communication, International Journal of Communication,
  International Journal of Press/Politics, Journalism and Mass
  Communication Quarterly, American Political Science Review
}

%\resumeitemize{2006 -- present}{Numerous Lectures and guest lectures, including}
%{
%\item Using Grammatical Analysis to determine Media Framing at Hebrew
% University (2013)
%\item Relational Content Analysis seminar at UT Austin and UT Pan American
%\item Content Analysis with Klaus Sch\"onbach at Friedrichshafen (2006, 2007, 2008), two sessions
%  each (with Nel Ruigrok)
%\item University of Amsterdam with (2009) in Research Master (with Nel Ruigrok)
%}

%\resumeitem{2001 -- 2004}{Numerous professional programming courses}

}

 

%%%%%%%%%%%%%%%%%%%%%%%%%%%%%%%%%%%%%%%%%%%%%%%%%%%%%%%%%%%%%%%%%%%%%%%%%%%%%%%%

\newcommand{\other}{
\section{Other relevant experience}

\resumeitem{2004 -- present}{AmCAT}
{Building and extending the open source Amsterdam Content Analysis Toolkit (AmCAT) 
for media storage, retrieval, annotation, and analysis; currently leading team of
4 developers for maintaining a system containing 40 million documents,
of which over 400 thousand manually annotated, in use by $>$100 users
at various institutions and companies. See
\url{http://github.com/amcat} and \url{http://amcat.nl}.
}

\resumeitemizeb{Research visits}
{\item July 2014: U. Vienna (hosted by Klaus Schoenbach), using AmCAT to study the Austrian elections 
 \item May 2006: UT Austin (hosted by Max McCombs), UT Pan American (hosted by
   Salma Ghanem), using semantic network analysis to study the US Elections
 \item 2009: Sciences-Po Paris (hosted by Emiliano Grossman), Agenda setting in French media
 \item 2004 -- 2008: Zeppelin U. Friedrichshaven (hosted by Klaus Schoenbach), automatic content analysis
}


}


%%%%%%%%%%%%%%%%%%%%%%%%%%%%%%%%%%%%%%%%%%%%%%%%%%%%%%%%%%%%%%%%%%%%%%%%%%%%%%%%

\newcommand{\money}{
  \section{Grants and acquisition}
  \resumeitem{2024}{ERC Horizon RIA}{WHAT-IF: Advanced Simulations for Testing the Effect of the Information Environment on the Functioning of Democracy (Lead PI; 3.2M, together with many others )}
  \resumeitem{2022}{Parliamentary Inquiry}{Context analysis for Parlementaire Enqu\^ete Fraudebeleid en dienstverlening (PEFD)}{PI, together with VU Law, 110k}
  \resumeitem{2021}{NWO SSH Platform Digital Infrastructure}{Digital Data Donation Infrastructure (D3I) (Co-PI; 1M, together with UvA and others)}
  
  \resumeitem{2020}{NWO OC Digitalisation}{Rethinking news algorithms, VU (PI; 750k\euro, together with VU linguistics and UvA law)}
  
  \resumeitem{2020}{VW AI \& Future Society Planning Grant}{Value-sensitive AI for augmenting local journalism (150k, co-applicant)}
  
\resumeitem{2020}{OPTED: H2020 Infrastructure grant}{Developing infrastructure for multilingual analysis of political text (WP PI; VU 285k\euro\ out of 3M\euro\ total)}
  
  \resumeitem{2018}{Dutch Jouralism Fund}{Local journalism and the municipal elections, VU 38k\euro out of 108k\euro total (PI: Nel Ruigrok)}
  
  \resumeitem{2017}{JEDS}{`Tracking the Filter Bubble', Joint E-science / Data Science grant, NWO \& Netherlands eScience Center, 250k\euro plus funding for eScience engineer (PI)}
  
\resumeitem{2017}{SIDN}{`Unlocking the Potential of News Recommenders for an Open Internet and Informed Citized', 75k\euro (PI: Natali Helberger)}

\resumeitem{2017}{Digging into Data}{``Responsible Terrorism Coverage (ResTeCo): A Global Comparative Analysis of News Coverage about Terrorism from 1945 to the Present'', co-PI with Scott Althaus and Hartmut Wessler (my part 125k\euro\ out of total 550k\euro), Digging into Data Challenge, Trans-Atlantic Platform for the Social Sciences and Humanities (NWO, NSF, DFG, others)}



\resumeitem{2017}{Anders Foundation}{The Hyperlinked News Network in Scandinavia, PI Helle Sjovaag (my part 40k\euro\ out of total 1M\euro), Anderstiftelsen (Anders Foundation), Sweden}


\resumeitem{2017}{Dutch Journalism Fund}{Media, audience, and elections: demand, supply, and effect of election coverage in a fractured political media landscape. Dutch Journalism Fund (www.svdj.nl; with Nel Ruigrok, Andr\'e Krouwel and Sara Gagenstein; 78k\euro)}

\resumeitem{2017}{Democracy and Media Foundation}{Discriminatory micro-portraits of Muslims in political coverage. Democracy and Media Foundation (www.stdem.org; with Nel Ruigrok and Antske Fokkens; 19k\euro)}


\resumeitem{2016}{Dasym}{Commissioned Research for Dasym Investment on monitoring social media and news media for investment decision support (with Piek Vossen and Nel Ruigrok; 49k\euro)}

\resumeitem{2014}{WODC}{Commissioned Research for Minitry of Justice on media coverage of youth crime (with Nel Ruigrok; 84k\euro)}

\resumeitem{2013}{WODC}{Commissioned Research for Minitry of Justice on trust in the rule of law (with Jan Kleinnijenhuis; 84k\euro)}

\resumeitem{2012 -- 2014}{NWO VENI grant}{Personal grant to study
  mediatized politics, see
  \url{http://vanatteveldt.com/pub/veni_atteveldt.pdf}  (250k\euro)}

\resumeitem{2012 -- 2014}{ERC FP7 grant e-com@EU}{Co-applicant of work
  package to study influence of media and social media coverage of the
  H1N1 pandemic (with Tilo Hartmann and Enny Das,
  in total 2M\euro, our work packages 250k\euro)}

\resumeitem{2013 -- 2014}{NI Academy Assistant Program}{Network Institute Grant
  for two `academy assistants' to kickstart interdisciplinary collaboration
   (with Laura Hollink; VU CS) and an additional Network Institute grant to
  hire an academy assistant to contribute to interdisciplinary
  collaboration on the Language Lab (with Antske Fokkens; VU Linguistics)}

\resumeitem{2012 -- 2014}{KNAW Academy Assistant Program}{Dutch Academy of Sciences Grant
  (administered via the Network Institute) for
  `academy assistants' to kickstart interdisciplinary collaboration
   (with Barbara Vis and Laura Hollink)}

\resumeitem{2012 -- 2013}{FSW Talent Program}{Selected for the VU Faculty Talent Program}

}


%%%%%%%%%%%%%%%%%%%%%%%%%%%%%%%%%%%%%%%%%%%%%%%%%%%%%%%%%%%%%%%%%%%%%%%%%%%%%%%%

\newcommand{\pub}{

\section{Publications}

\renewcommand{\refname}{Books}
\begin{refsection}[books.bib]
\nocite{*}
\footnotesize
\printbibliography[]
\end{refsection}

\renewcommand{\refname}{Journal Articles}
\begin{refsection}[articles.bib]
\nocite{*}
\footnotesize
\printbibliography[]
\end{refsection}


\renewcommand{\refname}{Book Chapters and other publications}
\begin{refsection}[other.bib]
\nocite{*}
\footnotesize
\printbibliography[]
\end{refsection}

}

%%%%%%%%%%%%%%%%%%%%%%%%%%%%%%%%%%%%%%%%%%%%%%%%%%%%%%%%%%%%%%%%%%%%%%%%%%

\details
\summary
\education
\work 
\money
\teaching 

%\other
\newpage
\pub 

\end{document}
